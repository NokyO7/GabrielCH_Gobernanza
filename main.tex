\documentclass{report}
\usepackage[spanish,es-noquoting]{babel}
\usepackage{graphicx}
\usepackage[sfdefault]{roboto}
\usepackage[T1]{fontenc}
\usepackage{geometry}  
    \geometry{left=24mm,right=24mm,top=24mm,bottom=24mm}
\usepackage[style=ieee]{biblatex}
    \addbibresource{bibliography.bib}
\usepackage{booktabs}
\usepackage{tikz}
\usepackage[text=Borrador \\ confidencial, fontsize=0.15\paperwidth, colorspec=0.96]{draftwatermark}
\usepackage{glossaries}
\usepackage{enumitem}

% Misceláneos
% -----------

% Iconos
\usepackage{fontawesome5}

% Para insertar texto auxiliar
\usepackage{lipsum}

% TikZ
\tikzset{
    container/.style={
        rectangle,
        rounded corners,
        draw=black,
        text width=8em,
        minimum height=3em,
        text centered},
    external/.style={
        rectangle,
        rounded corners,
        draw=black, dotted,
        text width=8em,
        minimum height=3em,
        text centered},
    gtfs/.style={
        rectangle,
        rounded corners,
        fill=gray!10,
        draw=black,
        minimum height=3em,
        text centered},
    user/.style={
        rectangle,
        rounded corners,
        draw=white,
        text width=4em,
        minimum height=2em,
        text centered},
    legend/.style={
        font=\footnotesize,
        text width=7em,
        draw=none},
}
% Configuración

\newcommand\acrdef[2][]{(\acrshort[#1]{#2}, \emph{\acrlong[#1]{#2}})}

% Acrónimos
\newacronym{cdr}{CDR}{Call Detail Records}
\newacronym{afc}{AFC}{Automated Fare Collection}
\newacronym{aresep}{ARESEP}{Autoridad Reguladora de los Servicios Públicos}
\newacronym{api}{API}{Application Programming Interface}
\newacronym{apc}{APC}{Automated Passenger Count}
\newacronym{arc-it}{ARC-IT}{Architecture Reference for Cooperative and Intelligent Transportation}
\newacronym{arima}{ARIMA}{Auto-Regressive Integrated Moving Average}
\newacronym{atis}{ATIS}{Advanced Traveler Information System}
\newacronym{atms}{ATMS}{Advanced Transportation Management Systems}
\newacronym{automl}{AutoML}{Automated Machine Learning}
\newacronym{avl}{AVL}{Automated Vehicle Location}
\newacronym{bccr}{BCCR}{Banco Central de Costa Rica}
\newacronym{bfo}{BFO}{Basic Formal Ontology}
\newacronym{bid}{BID}{Banco Interamericano de Desarrollo}
\newacronym{bp}{BP}{Backpropagation}
\newacronym{bptt}{BPTT}{Backpropagation Through Time}
\newacronym{cap}{CAP}{Common Alerting Protocol}
\newacronym{cps}{CPS}{Cyber Physical Systems}
\newacronym{cpss}{CPSS}{Cyber-Physical-Social Systems}
\newacronym{ctp}{CTP}{Consejo de Transporte Público}
\newacronym{dap}{DAP}{Data Acquisition and Processing}
\newacronym{dddas}{DDDAS}{Dynamic Data-Driven Applications Systems}
\newacronym{dl}{DL}{Deep Learning}
\newacronym{dot}{DOT}{U.S. Department of Transportation}
\newacronym{etsi}{ETSI}{European Telecommunications Standards Institute}
\newacronym{fnn}{FNN}{Feed-forward Neural Networks}
\newacronym{gis}{GIS}{Geographic Information System}
\newacronym{giz}{GIZ}{Deutsche Gesellschaft für Internationale Zusammenarbeit}
\newacronym{gps}{GPS}{Global Positioning System}
\newacronym{gtfs}{GTFS}{General Transit Feed Specification}
\newacronym{ict}{ICT}{Information and Communication Technologies}
\newacronym{incofer}{Incofer}{Instituto Costarricense de Ferrocarriles}
\newacronym{iot}{IoT}{Internet of Things}
\newacronym{ipts}{IPTS}{Intelligent Public Transportation Systems}
\newacronym{its}{ITS}{Intelligent Transportation Systems}
\newacronym{itss}{ITSS}{IEEE Intelligent Transportation Systems Society}
\newacronym{lidar}{LiDAR}{Laser Imaging Detection and Ranging}
\newacronym{lstm}{LSTM}{Long Short-Term Memory}
\newacronym{mae}{MAE}{Mean Absolute Error}
\newacronym{mape}{MAPE}{Mean Absolute Percentage Error}
\newacronym{mbta}{MBTA}{Massachusetts Bay Transportation Authority}
\newacronym{me}{ME}{Maximum Error}
\newacronym{medae}{MedAE}{Median Absolute Error}
\newacronym{mdip}{MDIP}{Mobility Data Interoperability Principles}
\newacronym{ml}{ML}{Machine Learning}
\newacronym{mopt}{MOPT}{Ministerio de Obras Públicas y Transportes}
\newacronym{mqtt}{MQTT}{Message Queuing Telemetry Transport}
\newacronym{mre}{MRE}{Mean Relative Error}
\newacronym{mse}{MSE}{Mean Square Error}
\newacronym{nas}{NAS}{Neural Architecture Search}
\newacronym{ngsi}{NGSI-LD}{Next Generation Service Interface with Linked Data}
\newacronym{netex}{NeTEx}{Network Timetable Exchange}
\newacronym{obe}{OBE}{On-Board Equipment}
\newacronym{odm}{ODM}{Origin-Destination Matrix}
\newacronym{ods}{ODS}{Objetivos de Desarrollo Sostenible}
\newacronym{onu}{ONU}{Organización de las Naciones Unidas}
\newacronym{trb}{TRB}{Transportation Research Board}
\newacronym{rmse}{RMSE}{Root Mean Square Error}
\newacronym{rnn}{RNN}{Recurrent Neural Network}
\newacronym{rtml}{RTML}{Real-Time Machine Learning}
\newacronym{siri}{SIRI}{Standard Interface for Real-time Information}
\newacronym{sitgam}{SITGAM}{Sistema Integrado de Transporte Público Masivo para la GAM}
\newacronym{slr}{SLR}{Systematic Literature Review}
\newacronym{smape}{SMAPE}{Symmetric Mean Absolute Percentage Error}
\newacronym{ssn}{SSN}{Semantic Sensor Network Ontology}
\newacronym{stdm}{STDM}{Spatio-Temporal Data Mining}
\newacronym{sos}{SoS}{System-of-Systems}
\newacronym{tcip}{TCIP}{American Transit Communications Interface Profiles}
\newacronym{tcm}{TCM}{Technology Compatibility Matrix}
\newacronym{tic}{TIC}{Tecnologías de Información y Comunicación}
\newacronym{tp}{TP}{Transporte Público}
\newacronym{uitp}{UITP}{Unión Internacional de Transporte Público}
\newacronym{v2x}{V2X}{Vehicle-to-Everything}
\newacronym{w3c}{W3C}{World Wide Web Consortium}
\newacronym{elt}{ELT}{Extract, Load, Transform}
\newacronym{kpi}{KPI}{Key Performance Indicator}
\newacronym{qos}{QoS}{Quality of Service}
\newacronym{eda}{EDA}{Event-driven Architecture}


\title{Diseño de sistemas de información para el transporte público inteligente \\ \large{Una guía práctica para tomadores de decisiones}}
\author{Escuela de Ingeniería Eléctrica \\ Universidad de Costa Rica}
\date{OCtubre 2025}

%%%%%%%%%%%%%%%%
\begin{document}
%%%%%%%%%%%%%%%%

\maketitle

%%%%%%%%
\section*{Resumen ejecutivo}
%%%%%%%%

\tableofcontents

% --------------------
\chapter{Introducción}
% --------------------

Los sistemas inteligentes de transporte público utilizan tecnologías de información y comunicación para facilitar la recolección y uso de datos masivos con diferentes objetivos y públicos meta \cite{visan2022towards}. Los \textit{sistemas de información}, por su parte, son estrategias y tecnologías que ofrecen canales de comunicación e información oportuna a las personas usuarias para utilizar eficazmente el servicio \cite{barfield1998human} antes, durante y después de un viaje. 

Este propuesta ofrece una arquitectura tecnológica y una estrategia organizacional para abordar de forma integral el problema de la disponibilidad de información para utilizar el servicio de transporte público en Costa Rica.

La propuesta busca responder preguntas como:

\begin{itemize}
    \item ¿Cómo ser compatibles con aplicaciones de planificación de viajes intermodales --como Google Maps, Moovit o Transit App-- y otros servicios que consumen y despliegan datos del transporte público en sitios web, aplicaciones o pantallas, ampliando así la oferta disponible para las personas usuarias?
    \item ¿Cómo crear un sistema de información que no esté limitado a unos componentes tecnológicos aislados sino uno que sea parte de una estrategia integral de atención y satisfacción de las personas usuarias?
    \item ¿Cómo maximizar el aprovechamiento de los datos recopilados, para que sean utilizados también para la operación (empresas concesionarias), gestión (CTP), planificación (MOPT), regulación (ARESEP) e investigación (academia y prensa) del servicio, entre múltiples usos posibles?
    \item ¿Cómo reducir los costos iniciales del sistema, habilitando una implementación gradual y ofreciendo alternativas de bajo costo para las etapas iniciales?
    \item ¿Cómo promover en la implementación del sistema la participación y competencia de distintos proveedores tecnológicos, así como la libertad de elegir los componentes mejor adaptados a las necesidades para evitar la monopolización innecesaria del servicio?
\end{itemize}

%%%%%%%%
\section{Objetivos del proyecto}
%%%%%%%%

%%%%%%%%
\section{Alcance del proyecto}
%%%%%%%%

Unidad Técnica de Transporte Público Inteligente
Propuesta Técnica y Financiera del Plan de Implementación

%%%%%%%%
\section{Beneficios y retorno de inversión}
%%%%%%%%

\begin{description}
    \item[Garantía contractual] Cualquier servicio puede ser brindado mediante un contrato de las partes interesadas con la Rectoría de la UCR, con una definición clara de plazos y entregables.
    
    \item[Flexibilidad de financiamiento] Por medio de la estructura administrativa propuesta (sección \ref{S:estructura}), la Unidad puede recibir financiamiento de distintas fuentes, promoviendo la agilidad de la ejecución. Inclusive, la propuesta incluye varias opciones de monetización que pueden ayudar a financiar parte de los servicios (sección \ref{S:monetizacion}).

    \item[Flexibilidad de implementación] permite la coexistencia con otros servicios y proveedores del mercado, facilitando una implementación gradual.

    \item[Soluciones a la medida] En coordinación estrecha con las autoridades nacionales, el sistema implementado puede satisfacer las demandas específicas de las partes interesadas.

    \item[Agilidad] Independencia de ciclos anuales de asignación de presupuestos.

    \item[Estandarización] La arquitectura de la propuesta técnica mostrada en la sección XYZ está basada en

    \item[Enfoque en seguridad y robustez] El sistema tiene un diseño orientado a la seguridad de los datos y la confiabilidad (ver sección \ref{S:riesgos}).

    \item[Consistencia y unicidad] Toda la información al público en distintos medios sería consistente entre sí y con los datos oficiales de la fuente única del CTP. El manejo unificado de todos los ``puntos de contacto'' de los usuarios con el sistema permite mejorar la experiencia del servicio (ver sección \ref{S:experiencia}).

    \item[Respaldo académico] Esta propuesta está hecha por la Escuela de Ingeniería Eléctrica en alianza con Lanamme y ProDUS.

    \item[Tiempo de lanzamiento reducido] 

    \item[Código abierto] 
    
\end{description}

%%%%%%%%
\section{Estructura de la unidad técnica}
%%%%%%%%
\label{S:estructura}

% -----------------
\chapter{Servicios}
% -----------------

La propuesta incluye la gestión de un ``sistema de información'' orientado a las personas usuarias, con la información oportuna para promover el uso del transporte público y mejorar la percepción de satisfacción con el servicio. Además, incluye el análisis de los datos del servicio.

%%%%%%%%
\section{Sistema de información}
%%%%%%%%

La oferta completa incluye los siguientes servicios:

\begin{enumerate}[noitemsep]
    \item Administración de datos abiertos y estandarizados del sistema de transporte público
    \item Desarrollo y gestión de medios \textit{electrónicos} para difusión de información, incluyendo:
    \begin{itemize}
        \item Página web
        \item Aplicaciones móviles (propias y de terceros)
        \item Pantallas informativas
    \end{itemize}
    \item Diseño de medios \textit{impresos} para difusión de información y para promoción, incluyendo:
    \begin{itemize}
        \item Rotulación (\textit{señalética}) en paradas y otros lugares
        \item Folletos para distribución y vallas informativas en exteriores
    \end{itemize}
    \item Gestión de los medios de atención al cliente
    \item Diseño de un sistema de identidad visual
    \item Gestión de campañas de comunicación en redes sociales
    \item Desarrollo e implementación de tecnologías de telemetría y rastreo de vehículos para información en tiempo real
\end{enumerate}

\subsection{Datos estáticos del servicio}

\subsection{Datos en tiempo real}

%%%%%%%%
\section{Divulgación de datos}
%%%%%%%%



%%%%%%%%
\section{Análisis de datos}
%%%%%%%%

Auditorías

%%%%%%%%
\section{Diseño de la experiencia de usuario}
%%%%%%%%
\label{S:experiencia}

\subsection{Atención al cliente}

\subsection{Identidad visual}

\subsection{Campañas de comunicación}

%%%%%%%%
\section{Monetización}
%%%%%%%%
\label{S:monetizacion}

\subsection{Publicidad en medios electrónicos}

\subsection{Uso del API}

\subsection{Análisis especializado de datos}

\subsection{Integración con servicios de movilidad}

\subsection{Información patrocinada}

% ----------------------
\chapter{Implementación}
% ----------------------

%%%%%%%%
\section{Metodología}
%%%%%%%%

La propuesta está desarrollada con base en la investigación sobre transporte público inteligente y sus estándares y arquitecturas tecnológicas, en la investigación en técnicas modernas de análisis de datos, en las buenas prácticas de diseño de software y en diseño de servicios en general. Para atender las preguntas generadoras es necesario comenzar con la adopción de ciertos principios de diseño, que son:

\begin{description}
    \item[Interoperabilidad] para ser compatibles con estándares y arquitecturas internacionales \cite{davies2019state} y facilitar en el presente y en el futuro la modernización y actualización del sistema.
    \item[Neutralidad tecnológica] para evitar el \textit{vendor lock-in} (uso restringido o propietario de una tecnología o servicio), en el cual un distribuidor de una marca tiene un control manipulador –a menudo extorsivo– de todo el sistema, lo cual es una ocurrencia típica en grandes sistemas informáticos en el país.
    \item[Unicidad del sistema] para evitar fraccionamiento de la información por causa de la multiplicidad de empresas concesionarias y la atomización de la administración del transporte público en el país, lo cual no debe ser un obstáculo para la calidad de la información para las personas usuarias.
\end{description}

La propuesta adopta los principios de interoperabilidad de datos de movilidad \acrdef{mdip} \cite{mdip2023modernizing} establecidos por una asociación de agencias de Estados Unidos líderes en transporte público inteligente. En esencia, estos principios promueven la interoperabilidad por medio del desarrollo, la adopción y la implementación generalizada de estándares abiertos, el acceso a herramientas con datos de movilidad de alta calidad, la posibilidad de seleccionar los componentes tecnológicos mejor adaptados a las necesidades y el empoderamiento del público general con datos de movilidad de calidad y bien distribuidos.


%%%%%%%%
\section{Arquitectura}
%%%%%%%%

La arquitectura 

\begin{figure}
\centering
\begin{tikzpicture}[node distance=4cm]
    \draw[dashed, gray] (-6, -8) -- (10, -8) node[anchor=south east, font=\tiny\scshape]{Producción de datos \faIcon{arrow-alt-circle-up}} node[anchor=north east, font=\tiny\scshape]{Consumo de datos \faIcon{arrow-alt-circle-down}};

    \node[container](obe){Equipo en \\ el vehículo \\\vspace{2mm} \faIcon{mobile} \faIcon{file-alt}};
    \node[user, right of=obe](operator){Conductor \\ \faIcon{user}};
    \node[container, below of=obe](realtime){Servidor en tiempo real \\\vspace{2mm} \faIcon{server} \faIcon{database} \faIcon{cogs} \faIcon{laptop} \faIcon{file-alt}};
    \node[legend, left of=realtime](legend){\textbf{Componentes} \\\vspace{2mm} \faIcon{server} Servidor \\ \faIcon{database} Base de datos \\ \faIcon{cogs} API \\ \faIcon{laptop} Administración \\ \faIcon{box-open} Paquete \\ \faIcon{chart-bar} Visualización \\ \faIcon{mobile} App móvil \\ \faIcon{hdd} Controlador \\ \faIcon{desktop} Navegador web \\ \faIcon{file-alt} Documentación};
    \node[container, right of=realtime](editor){Editor de \\ GTFS \textit{Schedule} \\\vspace{2mm} \faIcon{server} \faIcon{database} \faIcon{box-open} \faIcon{laptop} \faIcon{file-alt}};
    \node[user, right of=editor](agency){Administración \\ \faIcon{users}};
    \node[gtfs, below of=realtime](gtfs-rt){GTFS \textit{Realtime}};
    \node[gtfs, right of=gtfs-rt](gtfs-s){GTFS \textit{Schedule}};
    \node[container, below of=gtfs-rt](datahub){Concentrador \\ de datos \\\vspace{2mm} \faIcon{server} \faIcon{database} \faIcon{cogs} \faIcon{laptop} \faIcon{file-alt}};
    \node[external, right of=datahub](google){Google Maps};
    \node[external, right of=google](moovit){Moovit};
    \node[external, left of=datahub](data){Otros datos};
    \node[container, below of=datahub](screens){Servidor de pantallas \\\vspace{2mm} \faIcon{server} \faIcon{database} \faIcon{laptop} \faIcon{file-alt}};
    \node[container, right of=screens](website){Sitio web \\ del servicio \\\vspace{2mm} \faIcon{desktop} \faIcon{file-alt}};
    \node[external, right of=website](services){Otros servicios};
    \node[container, left of=screens](analysis){Servidor de análisis de datos \\\vspace{2mm} \faIcon{server} \faIcon{database} \faIcon{cogs} \faIcon{box-open} \faIcon{chart-bar} \faIcon{file-alt}};
    \node[container, below of=screens](screen){Pantalla en buses o paradas \\\vspace{2mm} \faIcon{hdd} \faIcon{desktop} \faIcon{file-alt}};
    \node[user, below of=analysis](researcher){Investigador \\ \faIcon{user}};
    \node[user, below of=website](passenger){Pasajero \\ \faIcon{user}};
    \node[container, below of=services](signage){Editor de señalética \\\vspace{2mm} \faIcon{box-open} \faIcon{file-alt}};

    \path[every node/.style={
            font=\tiny,
            text centered,
  		fill=white,
  		align=center,
        }, >=latex
    ]
        (operator) edge[->, dashed] (obe)
        (obe) edge[->] node{https / api \\ \faIcon{code}} (realtime)
        (agency) edge[->, dashed] (editor)
        (editor) edge[<->] (realtime)
        (editor) edge[->] node{.txt / .zip \\ \faIcon{table}} (gtfs-s)
        (realtime) edge[->] node{protobuf \\ \faIcon{shapes}} (gtfs-rt)
        (gtfs-rt) edge[->] node[near start]{https \\ \faIcon{file-download}} (moovit)
        (gtfs-rt) edge[->] node[near start]{https \\ \faIcon{file-download}} (google)
        (gtfs-rt) edge[->] node[near start]{https \\ \faIcon{file-download}} (datahub)
        (gtfs-s) edge[->] node[near end]{https \\ \faIcon{file-download}} (moovit)
        (gtfs-s) edge[->] node[near end]{https \\ \faIcon{file-download}} (google)
        (gtfs-s) edge[->] node[near end]{https \\ \faIcon{file-download}} (datahub)
        (data) edge[->] (datahub)
        (datahub) edge[->] node{https / api / rss / ws / ngsi-ld \\ \faIcon{rss}} (services)
        (datahub) edge[->] node{https / api \\ \faIcon{code}} (website)
        (datahub) edge[->] node{https / api \\ \faIcon{code}} (screens)
        (datahub) edge[->] node{https / api \\ \faIcon{code}} (analysis)
        (screens) edge[->] node{https / websocket \\ \faIcon{compress-alt}} (screen)
        (analysis) edge[<-, dashed] (researcher)
        (passenger) edge[->, dashed] node[]{\faIcon[regular]{eye}} (screen)
        (passenger) edge[->, dashed] node[]{\faIcon[regular]{eye}} (website)
        (passenger) edge[->, dashed] node[]{\faIcon[regular]{eye}} (services)
        (services) edge[->] node{https / api \\ \faIcon{code}} (signage)
    ;
\end{tikzpicture}
\caption{Diagrama tecnológico de la implementación del sistema de información}
\label{F:arquitectura}
\end{figure}

%%%%%%%%
\section{Recursos y presupuestos}
%%%%%%%%

\subsection{Software provisto}

La Universidad de Costa Rica ofrece en esta propuesta, sin costo adicional, el desarrollo de las siguientes piezas de software, valoradas en \$ 123456:

Infraestructura informática

Instalaciones

- Oficina de al menos 10 metros cuadrados

Personal

- Desarrollador \textit{senior}
- Dos desarrolladores \textit{junior}

Servicios profesionales

- Servicios de diseño gráfico

Equipo

Mobiliario


%%%%%%%%
\section{Riesgos y estrategias de mitigación}
%%%%%%%%
\label{S:riesgos}

\subsection{Seguridad informática}

%%%%%%%%
\section{Plan de seguimiento y evaluación}
%%%%%%%%

%%%%%%%%
\section{Cronograma}
%%%%%%%%

\appendix

%%%%%%%%
\chapter{Software desarrollado}
%%%%%%%%



%%%%%%%%%%%%%%%%%%
\printbibliography
%%%%%%%%%%%%%%%%%%

\end{document}
